
%----------------------------------------------------------------------------------------
%	PACKAGES AND OTHER DOCUMENT CONFIGURATIONS
%----------------------------------------------------------------------------------------

\documentclass[12pt]{article}
\usepackage[english]{babel}
\usepackage[utf8x]{inputenc}
\usepackage{amsmath}
\usepackage{graphicx}
\usepackage[colorinlistoftodos]{todonotes}
 \usepackage{url}
 \documentclass{article}
\usepackage{xcolor}
\usepackage{titlesec}
\titleformat{\section}[block]{\color{blue}\Large\bfseries\filcenter}{}{1em}{}

 
\begin{document}

\begin{titlepage}

\newcommand{\HRule}{\rule{\linewidth}{0.5mm}} 

\center 
 \includegraphics[width=0.4\columnwidth]{concordia-logo.png} 

\textsc{}  \\[2cm]
\textsc{\LARGE Concordia University}\\[1.5cm] 
\textsc{\Large Deliverable 2}\\[0.5cm] 
\textsc{\large }\\[0.5cm] 

{ \huge \bfseries EULER NUMBER}\\[0.5cm]
{ \normalsize \bfseries ETERNITY:NUMBERS}\\[1cm]

 \noident \textbf{Repository Address: }{\url{https://github.com/ZinniaRana/Eternity_Numbers/tree/master/Deliverable2}}\\[2cm]

\begin{minipage}{0.4\textwidth}
\begin{flushleft} \large
\text Submitted By:\\
Zinnia Rana \\
40074965
\end{flushleft}
\end{minipage}
~
\begin{minipage}{0.4\textwidth}
\begin{flushright} \large
Supervisor \\
Prof. Pankaj \text{Kamthan} 
\end{flushright}
\end{minipage}


\end{titlepage}


\section*{ABSTRACT} % Unnumbered section
\raggedright From previous analysis of the interviews and what possible domain model and its Use Cases can be, it helped me focus on the main aspects of the User Stories. I did some research and figured out the requirements of user for the calculator and estimated the priority and time considering time constraints and implementation factors. The user stories are source from existing user stories and mainly external existing user stories. Below are the details of the user stories which helped me implement my calculator important features.\\[2cm]

\section*{USER STORIES} \\  % Unnumbered section
%------------------------------------------------

\subsection{User Story 1} % Unnumbered section
\noindent \textbf {Statement: }As a Statistical Analyst, I want to calculate value of e base x to compute time taken to get a task done which will help me depict the graph distribution. \\*
\newline 
\noindent \textbf {Constraints:}  Accuracy, Reusability\\*
\newline
\textbf{Acceptance Criteria: } 
\begin{itemize}
\item Correct precise value of e is returned
\item Operations performed upon e must give precise answer upto 3 decimal places
  \item Time calculated must be precised upto 2 digits
\end{itemize}

\noindent \textbf {Priority: }Medium\\[0.4cm]
\newline
\noindent \textbf {Estimate: } 3
\newline





%------------------------------------------------
\newpage
\subsection{User Story 2} % Unnumbered section

\noindent \textbf {Statement: }As a research assistant, I want to save the result to use it in later calculations. \\*
\newline 
\noindent \textbf {Constraints:}  Re-usability\\*
\newline
\textbf{Acceptance Criteria: } 
\begin{itemize}
  \item Given any number pressed or result calculated, when press MM key, number saved in the memory should be equal to number displayed on screen.\newline

\end{itemize}

\noindent \textbf {Priority: }High\\[0.4cm]
\newline
\noindent \textbf {Estimate:} 1\\[2cm]



%------------------------------------------------
\subsection{User Story 3} % Unnumbered section
\noindent \textbf {Statement: }As a research assistant, I want to calculate growth rate so that I can estimate software utilization over the period of time. \\*
\newline 
\noindent \textbf {Constraints:} Precision, Usability\\*
\newline
\textbf{Acceptance Criteria: } 
\begin{itemize}
  \item Given that user press 'Op' to calculate  growth formula.
  \newline Then it prompts the user for input values in the equation.
  \newline And then system evaluates the equation and displays the result.\newline
\end{itemize}

\noindent \textbf {Priority:} High\\[0.4cm]
\newline
\noindent \textbf {Estimate:} 2\\*
\newline


%------------------------------------------------
\newpage
\subsection{User Story 4} % Unnumbered section

\noindent \textbf {Statement: }As a Student, I want predefined operations in calculator so that I can just select which equation to calculate and input values to get final output \\*
\newline 
\noindent \textbf {Constraints:} Re-usability\\*
\newline
\textbf{Acceptance Criteria: } 
\begin{itemize}
  \item Check the entered number for calculation is valid
  \item Check the expected output is accurate as expected.
  
\end{itemize}

\noindent \textbf {Priority:}High\\[0.4cm]
\newline
\noindent \textbf {Estimate:} 2\\[2cm]


%------------------------------------------------
\subsection{User Story 5} % Unnumbered section

\noindent \textbf {Statement: }As an Entrepreneur, I want to calculate probability of a particular instance occurring using Euler identity  so that I can estimate the priority of implementing features. \\*
\newline 
\noindent \textbf {Constraints:}  Precision, Extensibility\\*
\newline
\textbf{Acceptance Criteria: } 
\begin{itemize}
  \item Output result needs to be checked whether it is precise.
  \item Need to check correct precise value of pi and e is returned.
  
\end{itemize}

\noindent \textbf {Priority:}Medium\\[0.4cm]
\newline
\noindent \textbf {Estimate:} 4\\*
\newline

%------------------------------------------------
\newpage
\subsection{User Story 6} % Unnumbered section

\noindent \textbf {Statement: }As a Graduate Mathematics student, I want my calculator should contain e and pi to calculate Euler identity for probability solving questions. \\*
\newline 
\noindent \textbf {Constraints:}  Extensibility, Re-usability\\*
\newline
\textbf{Acceptance Criteria: } 
\begin{itemize}
  \item Correct precise value of pi returned.
  \item Operations feasible with pi and other mathematical constants.
  \item Result displayed should match precision decimal points user entered.\newline
\end{itemize}

\noindent \textbf {Priority:}Medium\\[0.4cm]
\newline
\noindent \textbf {Estimate:} 5\\*
\newline
\section*{BACKWARD TRACEABILITY MATRIX}  % Unnumbered section
\newline
\centering
\begin{tabular}{ |p{3cm}||p{2cm}|p{2cm}|p{2cm}|p{2cm}|  }

 \hline
 
  & Use Case & User Story & Interview & External\\[0.5cm] \hline
 \hline

User Story 1    &   &   X &   &  \\[0.5cm] \hline
User Story 2    &   & UC 1  &   &\\[0.5cm] \hline
User Story 3    &   &   &   & X \\[0.5cm] \hline
User Story 4    & X  &   &   & \\[0.5cm] \hline
User Story 5    &   &   &   & X \\[0.5cm] \hline
User Story 6    &  US 5 &   &   & \\[0.5cm] \hline
 \hline
\end{tabular}
\newpage
\section*{CONCLUSION}  % Unnumbered section
\raggedright After analyzing the user requirements and what features they want to be implemented in the  calculator, I have chosen three  highest priority  user stories for implementing my calculator features, keeping time constraints into account.
\section*{REFERENCES}  % Unnumbered section

\begingroup
\renewcommand{\section}[2]{}%
\begin{thebibliography}{}
\bibitem{1} 
https://betterexplained.com/articles/an-intuitive-guide-to-exponential-functions-e/
 
\bibitem{2} 
https://www.quora.com/Can-someone-explain-the-number-e-Eulers-number-to-me
 
\bibitem{3} 
https://medium.com/@ozanerhansha/applications-of-eulers-formula-857bf60ba32d
\bibitem{4}
https://www.slideshare.net/DhavalDalal/calculator-stories
\bibitem{5}
https://www.purplemath.com/modules/expofcns5.htm
\end{thebibliography}
\end{document}
